\section{Revisão Sistemática de Literatura}

\begin{center}
  \vspace{1cm}
  27 de julho às 10h30m, Revisão Sistemática de Literatura, Katia Romero Felizardo, Professora na UTFPR-Cornélio Procópio.
  \vspace{1cm}
\end{center}

A oradora iniciou a apresentação do conteúdo chamando a atenção para a importância da revisão da literatura de forma sistemática, apresentando assim, os principais benefícios proporcionados, os quais serão discorridos na sequência. Inclusive, tornou evidente que é um protocolo recente na área de computação, em que algumas etapas foram estabelecidas a partir do estudo de outras áreas, como, por exemplo, a medicina.

O primeiro tópico abordado foi a respeito dos estudos secundários, os quais sumarizam os estudos primários, através de Mapeamento Sistemático (MS) e/ou Revisão Sistemática (RS). Os estudos primários (estudos de caso, experimentos controlados, surveys) podem ser encontrados em fontes de busca como: IEEE Xplore, Scopus e ACM, por exemplo. E, para buscar esses artigos, é possível utilizar uma string de busca, a qual contém palavras chave da área de pesquisa a ser investigada.

Uma característica importante a ser diferenciada nos componentes do estudo secundário é o escopo. Sendo que no MS é mais genérico do que na RS. Ou seja, no MS tem-se uma visão ampla do tópico de pesquisa. Enquanto que na RS busca-se identificar, selecionar, avaliar, interpretar e sumarizar os estudos primários considerados relevantes no tópico de pesquisa. Assim, um MS pode ser feito antes da RS, e esta, por sua vez, complementar de forma mais detalhada o mapeamento inicial.

A partir da compreensão da importância do MS e da RS, inicia-se a fase de planejamento. Nessa fase, o objetivo é definir o protocolo e posteriormente avaliá-lo. Na definição cria-se um plano predefinido que formaliza todo o processo para a execução de uma RS ou de um MS. Esse protocolo tem como objetivo principal a redução de vieses ou ambiguidades que possam ocorrer durante a execução do MS ou da RS. O protocolo pode ser estruturado nas seguintes seções:

\begin{itemize}
  \item ``Informações gerais'': que são as informações do título do mapeamento, os pesquisadores, a descrição e os objetivos.
  \item ``Questões de pesquisa'': são questões que quando respondidas ajudam a alcançar os objetivos do Mapeamento Sistemático.
  \item ``Identificação de estudos'': são as palavras chave, a string de busca, os critérios de seleção das fontes de busca, lista das fontes de busca e estratégia de busca.
  \item ``Seleção e avaliação dos estudos'': são os critérios de inclusão e de exclusão dos estudos primários, estratégia para a seleção e avaliação da qualidade desses estudos.
  \item ``Síntese dos dados'': síntese dos dados e apresentação dos estudos. Nesta seção estão as informações sobre a estratégia de extração de dados, estratégia de sumarização dos dados e estratégia de publicação.
\end{itemize}

A seleção dos estudos pode ser realizada através de critérios. Os critérios de inclusão ajudam a incluir estudos relevantes. E os critérios de exclusão descartam os estudos que não são de interesse, ou seja, que não serão usados para responder às questões de pesquisa.

Por fim, verifica-se a validade do protocolo, a qual deve ser realizada por meio do teste do protocolo, chamado de teste piloto. O objetivo do teste é verificar a viabilidade de execução do mapeamento, permitindo também, com base nos resultados do teste, identificar modificações que sejam necessárias.
A oradora finalizou a apresentação com exemplos reais em que a string de busca e as questões de pesquisa não foram bem estabelecidas, o que impactou em negativas de publicação e retrabalho para realizar todo o MS e RS novamente. Assim, destacou-se a importância de ater-se ao protocolo, executando todas as suas etapas e validando-as.
