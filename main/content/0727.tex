\section{Revisão Sistemática de Literatura}

\begin{center}
  \vspace{1cm}
  27 de julho às 10h30m, Revisão Sistemática de Literatura, Katia Romero Felizardo, Professora na UTFPR-Cornélio Procópio.
  \vspace{1cm}
\end{center}

Inicia-se a apresentação destacando a importância na realização da revisão da literatura de forma sistemática. Apresenta-se os principais benefícios na execução destas técnicas e argumenta-se que este é um protocolo ainda recente na área da computação, algumas etapas foram estabelecidas com base no estudo em outras áreas, como por exemplo a medicina. 

O tópico relaciona aos estudos secundários foi o primeiro a ser abordado. Seu principal objetivo é sumarizar os estudos primários através de um Mapeamento Sistemático ou uma Revisão Sistemática. Os estudos primários (surveys, estudos de caso ou experimentos controlados, por exemplo) podem ser encontrados em repositórios de busca como: IEEE Xplore, Scopus ou ACM, por exemplo. Mostra-se ainda a importância na criação e no refinamento de uma string de busca, que é formada por uma composição de palavras chave com relação ao tema de pesquisa.

O escopo é uma característica a ser direfencia no escopo secundário. Argumenta-se que um Mapeamento Sistemático é uma exploração mais genérica que uma Revisão Sistemática. No Mapeamento Sistemático aplicada uma visão mais ampla do assunto de pesquisa, enquanto que na Revisão Sistemática, o intúito é identificar, classificar, selecionar, avaliar, interpretar e sumarizar os estudos primários considerados relevantes no tópico de pesquisa. Desta forma, recomenda-se que um Mapeamento Sistematico seja feito antes da Revisão Sistemática, que deve complementar de forma mais detalhada o mapeamento realizado inicialmente.

Após a compreensão da inportância do Mapeamento e Revisão sistemática, deve-se iniciar a fase de planejamento. O objetivo é definir um protocolo e posteriormente avaliá-lo. Seu principal objetivo é a redução de vieses ou ambiguidades que possam ocorrer durante a execução do MS ou da RS. O protocolo pode ser estruturado nas seguintes seções:

\begin{itemize}
  \item ``Informações gerais'': título do mapeamento, autores, a descrição e objetivos;
  \item ``Questões de pesquisa'': questões que ajudam a alcançar os objetivos do Mapeamento Sistemático;
  \item ``Identificação de estudos'': palavras chave, a string de busca, os critérios de seleção, lista das fontes de busca e estratégia de busca;
  \item ``Seleção e avaliação dos estudos'': critérios de inclusão e de exclusão dos estudos primários, estratégia para a seleção e avaliação da qualidade desses estudos;
  \item ``Síntese dos dados'': síntese dos dados e apresentação dos estudos.
\end{itemize}

Os estudos podem ser selecionados através de critérios que ajudam filtrar quais são os mais relevantes. Critérios de exclusão descartam estudos que não são de interesse para o escopo do projeto. No final, deve-se verificar a validade do protocolo por meio de um teste chamado piloto. Este teste deve verificar a viabilidade de execução do mapeamento, permitindo ainda, com base nos resultados, identificar possíveis modificações ou ajustes caso sejam necessários.

Finaliza-se a apresentação de exmplos reais em que o resultado esperado não foi obtido pelas questões de pesquisa não estarem bem estabelecidas e pela string de busca não ter sido refinada o suficiente. Destacando-se a importância de seguir o protocolo, executando todas as suas etapas e validando-as.
