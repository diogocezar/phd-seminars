\section{Inclusividade e Comportamento em Comunidades Open Source}

\begin{center}
  \vspace{1cm}
  10 de agosto às 13h30m, Inclusividade e Comportamento em Comunidades Open Source, Igor Steinmacher, Professor na Northern Arizona University, e na UTFPR-Campo Mourão. https://bbb.c3sl.ufpr.br/b/lui-gv2-9xx
  \vspace{1cm}
\end{center}

O orador começou abordando a dificuldade de entrada em grandes projetos, considerando a necessidade de aprendizado de forma rápida. Além disso, a alta rotatividade de pessoas nas equipes de desenvolvimento dificulta a manutenção de um trabalho contínuo. Assim, o orador ressaltou que a engenharia de software é um ato social, demonstrando as contribuições de programadores no desenvolvimento do Mozilla Firefox.

Na sequência o Professor continuou abordando a relação social entre os desenvolvedores, demonstrando que estes participam de mais de uma comunidade de software open source. Apresentou também a comunidade OpenStack, a qual possui as seguintes características: 1.7 milhão de linhas de código, 19 linguagens de programação, 17 mil membros, 4,5 mil contribuidores de código, 38 mil mensagens de e-mail e 51 mil seguidores no Twitter.

O orador continuou ressaltando a importância do software livre  na indústria, com os seguintes dados (de acordo com um survey de 2017): 60\% das organizações usam código aberto, o impacto do software livre nas organizações foi de 55\% de aumento na inovação e 44\% na qualidade, e 66\% dessas organizações também contribuíram com projetos de código aberto. Algumas dessas companhias são: Apple, Google e Microsoft.

A palestra continuou com o software de código aberto voltado para a academia. O orador comentou que atualmente não é mais necessário desenvolver um sistema, mas sim reutilizar projetos que já se encontram disponíveis. Assim, o palestrante citou que os softwares de código aberto podem ser utilizados para o ensino em disciplinas, por exemplo, avaliação de interfaces ou refatoramento de código. E que, em alguns casos, os alunos chegaram a ser chamados para contribuir com projetos open source.

O orador abordou o objetivo da pesquisa realizada por ele. Demonstrou a metodologia utilizada: estudos empíricos, engenharia e avaliação. Assim, primeiramente há a necessidade de mineração dos repositórios de software, buscando entender a interação entre os membros da comunidade (commits, processos, comportamento, ferramentas, etc.). E, em um segundo momento, faz-se necessário o uso de um método qualitativo, buscando entender a socialização entre os membros da comunidade de forma subjetiva.

Na sequência o palestrante ressaltou o problema central que desenvolvedores enfrentam ao tentar participar da comunidade de código aberto: documentação. Citou que existem outros problemas como código legado, comunicação, etc. Mas a documentação mal estruturada foi o ponto chave que dificulta a entrada de novos membros na comunidade. Assim, o Professor fez um experimento tentando entender melhor como superar as barreiras levantadas e propôs um portal para estruturar a documentação e facilitar o entendimento desta.

Outro assunto abordado foi o comportamento em vários projetos de software de código aberto. Percebeu-se que em vários destes, mais de 50\% dos commits eram únicos e feitos por uma pessoa. Ressaltou-se que apesar de ser um commit único, eram funcionalidades inteiras desenvolvidas e não apenas modificações simples de texto, por exemplo. Ademais, percebeu-se que vários commits eram ignorados pela comunidade.

Por fim, o palestrante citou um projeto que criou para gamificação em desenvolvimento de software livre. O intuito era ensinar o passo a passo para entender a documentação, a arquitetura e como contribuir com o desenvolvimento. E finalizou ressaltando que leciona apenas com software livre e que está desenvolvendo uma ferramenta para facilitar a recomendação de projetos para os desenvolvedores, a qual através do perfil sugere software de código aberto que pode melhorar as habilidades deste desenvolvedor.