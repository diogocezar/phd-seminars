\section{Inclusividade e Comportamento em Comunidades Open Source}

\begin{center}
  \vspace{1cm}
  10 de agosto às 13h30m, Inclusividade e Comportamento em Comunidades Open Source, Igor Steinmacher, Professor na Northern Arizona University, e na UTFPR-Campo Mourão. https://bbb.c3sl.ufpr.br/b/lui-gv2-9xx
  \vspace{1cm}
\end{center}

A apresentação se inicia com o orador demostrando a dificuldade de ingresso em projetos grandes, com o aspecto de uma necessidade de aprendizado rápido, e também como a mudança de equipe constantemente ocasiona uma maior dificuldade para se realizar em trabalho de forma contínua. Houve também o apontamento do orador dizendo que a engenharia de software é um ato social, e também falando das contribuições dos programadores no desenvolvimento o Mozilla Firefox.

O Professor, contúdo, abordou o modo de relacionamento social dos desenvolvedores, mostrando que houvesse mais a participação, dos mesmos, uma comunidade de Open Source, e também fez a apresentação da comunidade de Open Stack, que agrega: 1,7 milhões de linhas de código, 19 linguagens de programação, 17 mil membros, 4,5mil contribuintes de código, entre outros.

Na palestra o orador expos que não há mais a necessidade de desenvolver um sistema por completo, mas que já é possível reutilizar códigos já disponíveis. Assim os códigos abertos podem ser utilizados no meio acadêmico, como forma de ensino para disciplinas.

Ainda na palestra o orador falou de seu método utilizado, sendo eles o aspecto empírico, engenharia e avaliação, e em um segundo momento utilizou-se do método qualitativo, visando entender o modo subjetivo em que os membros da comunidade se socializavam.

Houve também a problematização da utilização do código aberto, como a documentação, que se é formulada com uma má estrutura dificulta a entrada de novos membros no projeto.

Assim, o palestrante falou que houve a criação de um projeto para dar assistência a no desenvolvimento de software livre, e que proporcionava ajudar a entender a documentação, a arquitetura e a contribuição para criar o desenvolvimento.