\section{Xadrez e Educação}

\begin{center}
  \vspace{1cm}
  17 de agosto às 10h30m, Xadrez e Educação, Prof. Wilson Silva, CEX, https://bbb.c3sl.ufpr.br/b/lui-gv2-9xx
  \vspace{1cm}
\end{center}

O seminário apresentado pelo Prof. Dr. Wilson da Silva, iniciou-se pela apresentação de seu currículo, no qual destaca-se trabalhos relacionados ao Xadrez com foco na educação, e livros de sua autoria direcionados à divulgação e ao aprendizado do xadrez. Foram três principais tópicos abordado durante a apresentação: uma breve apresentação histórica do xadrez, a correlação entre o xadrez e a inteligência artificial e o xadrez relacionado a educação.

Pela linha do tempo do xadrez, apresentada pelo Prof. Wilson, nota-se que o esporte tem início por volta de 500 d.C. quando os primeiro jogos, bem diferentes do xadrez convencional, deram início aos estudos desta área. Entre evoluções e convergências para o jogo que temos hoje, chega-se às competições entre humanos e computadores.

O primeiro jogo apresentado, chamado de Chaturanga, deu foi o precursor do Xadrez, neste jogo, destaca-se a possibilidade de jogo entre até 4 jogadores e a presença da peça "Elefante", que inicialmente podia mover-se para a segunda casa na diagonal em que se encontrava, saltando a casa intermediária, se estivesse ou não ocupada. O Elefante foi a primeira derivação do bispo, que conhecemos nos jogos mais modernos de Xadrez.

Ao se tratar de da correlação entre o Xadrez e a inteligência artifical, uma linha do tempo foi apresentada em que destaca os principais acontecimentos. Em 1770 um autômato chamado de O Turco deu início a uma máquina que jogava Xadrez. Destaca-se ainda em 1956 o primeiro programa de xadrez executado por um computador. Este programa, executava uma versão mais simplificada do Xadrez, e era executado pelo computador MANIAC I. Em 1996, o Deep Blue foi o primeiro proama de xadrez a vencer uma parteida para um campeão mundial (Kasparov). Finalmente, em 1997 o Deep Blue vence o campeão mundial em um match. Atualnente, existem competições apenas entre computadores, que tem o objetivo de desafiar as inteligências desenvolvidas por programadores.

Na sequência foi enfatizada a importância do xadrez na educação e apresentado o trabalho ``Xadrez Livre'', uma estrutura na Internet própria para permitir e estimular o jogo de Xadrez. Em 2001 foi implementado o primeiro servidor na UFPR, utilizando ainda a arquitetura FICS. Em 2008, um novo programa foi implementando o Xadres Livre.

Além disso, ainda nos trabalhos executados na UFPR, foi destacado o trabalho de FEITOSA, 2013, p.8. Uma ferramenta chamada de HeuChess que analisa aspéctos das partidas como: espaço (tabuleiro), tempo (lances), matéria (peças), segurança do rei, tática e estratégia.

Na sequência falou-se sobre o Xadrez e seu impácto direto na Educação. O jogo pode se aprodundar em aspéctos: educacionais, científicos, artísticios/culturais e desportivos. Do ponto de vista cognitivo, pode melhorar: memória, raciocínio lógico e matemático, abstração, resolução de problemas e concentração. Além disso, o jogo pode: aumentar a integração social, melhorar a autoestima, permitir a inclusão social, ser uma atividade extracurricular atrativa. Tudo isso com uma ótima relação de custo/benefício.

Destacou-se ainda que o Xadrez é um esporte. A FIDE (International Chess Federation), fundada em 1924, foi reconhecia em 1999 pelo Comitê Olímpico Internacional como uma Federação Internação do Desporto.

Outro destaque foi como o Xadrez é inclusivo, permitindo que qualquer pessoa, independente de suas necessidades especiais, possam jogar, treinar e participar de competetições. Além disso, Homens, Mulheres, Adultos e Crianças, podem competir em condições iguais.

Seguindo na apresentação, apresentou-se algumas metodologias para o ensino do Xadez. No quadro com Jogos Pré-Enxadrísticos e objetivos, destaca algumas variações do jogo que podem ser aplicadas para iniciação e conhecimento do jogo.

Já no quadro com Partidas Temáticas, mostra-se algumas partidas que podem ser executadas sem todas as peças do tabuleiro. No Quadro Método Holandês, demonstra-se possibilidades de treinamento no qual um dos jogadores deve jogar apenas com o Rei, enquanto que o outro jogador tem uma coleção de peças que pode variar (rainha, torre, bispo, cavalo e peões) com o objetivo de chegar a um cheque-mate.

O Xadrez é um ótimo caminho para o aprimoramento mental, social e competitivos. Mas como todo esporte, exige treinamento, dedicação e estudo. Durante toda a evolução do jogo, até se tornar um esporte, grandes enxadristas deixaram sua assinatura, entretando, mais recentemente o poder computacional, e a possibilidade de previsão possíveis com programas de computadores, desafiaram a capacidade humana. Atualmente, o Xadrez não é mais disputado entre Humanos e Computadores, pois a desvantagem se tornou expressiva.