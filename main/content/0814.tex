\section{A computação e a pesquisa aplicada no Combate ao COVID}

\begin{center}
  \vspace{1cm}
  14 de agosto às 18H00m, A Computação e a pesquisa aplicada no Combate ao COVID, Painel 5 UFF, http://ev-ppgc.ic.uff.br/2020-2/paineis.html
  \vspace{1cm}
\end{center}

A palestra iniciou com a Doutora Cristina Fontes do Hospital Universitário Antônio Pedro e Professora na Universidade Federal Fluminense (UFF), a qual falou sobre o trabalho que realiza no hospital e na UFF. Um dos projetos citados é a pesquisa realizada de tomografia computadorizada do tórax nos profissionais de saúde expostos a COVID-19. A oradora ainda comentou sobre o protocolo que seguem no hospital através de características das lesões, topografia destas, derrame pleural, achados mediastinais e outros achados pulmonares.

A segunda palestrante foi a Doutora Paula Santos da Universidade de São Paulo (USP), a qual iniciou a apresentação comentando sobre o sistema para diagnóstico de COVID-19 que está em desenvolvimento, chamado “Marie”. Assim, a oradora explicou as etapas que seguiram de pesquisa para a criação do sistema, como: obtenção dos dados, definição de métodos (PCA), construção de interface e disponibilização. A oradora finalizou explicando as dificuldades envolvidas na elaboração do método, como: entendimento da área médica (em relação às características das imagens), falta de imagens de exemplos/validação e definição do modelo matemático.

O próximo palestrante a apresentar foi o Doutor Fabio Porto do Laboratório Nacional de Computação Científica (LNCC), o qual iniciou descrevendo os trabalhos realizados pelo laboratório Data Extreme Lab (DEXL) que pesquisa na área de Inteligência Artificial, Ciência dos Dados e Big Data. De forma mais detalhada, o orador comentou sobre o trabalho que estão realizando em relação às pesquisas sobre COVID, sendo que a filtragem de fake news é um tema bem relevante neste momento em que há uma grande quantidade de informações e pesquisas sendo publicadas. O orador ainda comentou sobre os sistemas que estão desenvolvendo como o monitorador de contatos, para avisar as pessoas que tiveram contatos com positivados para a COVID.

O palestrante continuou explicando sobre um trabalho que envolveu vários pesquisadores do LNCC, no qual identificaram genomas mostrando introdução e dispersão geográfica da transmissão do vírus no Brasil. O estudo ainda indicou a evolução da epidemia e os efeitos do distanciamento social no Brasil. Ainda foi comentado sobre o estudo estatístico desenvolvido para prever a necessidade de leitos por dia de acordo com o formato de distanciamento social, sendo que o detalhamento foi feito para cada estado da federação.

O orador ainda comentou sobre a ferramenta desenvolvida no LNCC, chamada DockThor. Essa ferramenta permite modelar proteínas para o desenho de fármacos em busca do entendimento do vírus SARS-CoV-2. Ainda finalizou apresentando a ferramenta PcDars em parceria com a Fio Cruz, que possibilita o monitoramento da COVID para verificar a evolução da pandemia.

Por fim, os palestrantes comentaram sobre as relações interdisciplinares da medicina com a computação, as novas aberturas e os desafios. Na sequência, também responderam perguntas acerca das ferramentas de detecção de COVID (raio-x e ultrassom), estudos recentes sobre pacientes recuperados e privacidade envolvendo os aplicativos desenvolvidas para atender a população (voltados a COVID).
