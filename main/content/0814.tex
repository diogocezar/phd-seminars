\section{A computação e a pesquisa aplicada no Combate ao COVID}

\begin{center}
  \vspace{1cm}
  14 de agosto às 18H00m, A Computação e a pesquisa aplicada no Combate ao COVID, Painel 5 UFF, http://ev-ppgc.ic.uff.br/2020-2/paineis.html
  \vspace{1cm}
\end{center}

A doutora Cristina Fontes, do Hospital Universitário Antônio Pedro e professora na Universidade Federal Fluminense (UFF), iniciou sua fala sobe o trabalho que realiza no hospital e na UFF. Citou a pesquisa realizada de tomografia computadorizada do tórax nos profissionais de saúde expostos a COVID-19. Salientou ainda, sobre o protocolo que seguem no hospital através de características das lesões, topografia destas, derrame pleural, achados mediastinais e outros achados pulmonares.

Em seguida, palestrou a doutora Paula Santos da Universidade de São Paulo (USP), ela falou sobre o sistema para diagnóstico de COVID-19 que está em desenvolvimento, chamado “Marie”, explicou as etapas que seguiram de pesquisa para a criação do sistema e também as dificuldades envolvidas na elaboração do método. 

Em sequência, apresentou o Fabio Porto do Laboratório Nacional de Computação Científica (LNCC), descrevendo os trabalhos realizados pelo laboratório Data Extreme Lab (DEXL) que pesquisa na área de Inteligência Artificial, Ciência dos Dados e Big Data. O palestrante abordou sobre o trabalho que estão realizando em relação às pesquisas sobre COVID. Além disso, comentou sobre os sistemas que estão desenvolvendo como o monitorador de contatos, para avisar as pessoas que tiveram contatos com positivados para a COVID.

Fabio Porto, ainda explicou sobre um trabalho que envolveu vários pesquisadores do LNCC, no qual identificaram genomas mostrando introdução e dispersão geográfica da transmissão do vírus no Brasil. Este estudo indicou a evolução da epidemia e os efeitos do distanciamento social no Brasil. Comentou sobre o estudo estatístico para prever a necessidade de leitos por dia de acordo com o formato de distanciamento social.

O palestrante discorreu sobre a ferramenta desenvolvida no LNCC, chamada DockThor e finalizou apresentando a ferramenta PcDars em parceria com a Fio Cruz. 
Por último, os palestrantes falaram sobre as relações interdisciplinares da medicina com a computação, as novas aberturas e os desafios, em seguida responderam perguntas.
