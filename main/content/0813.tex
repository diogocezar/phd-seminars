\section{Aplicações e Desafios de Pesquisa na Computação em Nuvem e Alto Desempenho}

\begin{center}
  \vspace{1cm}
  13 de agosto às 18H00m, Aplicações e Desafios de Pesquisa na Computação em Nuvem e Alto Desempenho, Painel 4 UFF, http://ev-ppgc.ic.uff.br/2020-2/paineis.html
  \vspace{1cm}
\end{center}

A apresentação começou com a mediação da Professora Doutora Lúcia Drummond da Universidade Federal Fluminense (UFF), a qual definiu computação em nuvem como: paradigma de computação distribuída de larga escala, onde recursos computacionais estão disponíveis aos usuários através da Internet. A oradora também elencou os tipos de serviços em nuvem: IaaS (Infrastructure as a Service), SaaS (Software as a Service), PaaS (Platform as a Service) e FaaS (Function as a Service).

Na sequência, a oradora abordou aplicações que demandam uso de supercomputadores, como: mecânica quântica, previsão de tempo, exploração de óleo e gás, modelagem molecular e simulações físicas. Assim, ressaltou a importância que HPC (High Performance Computing) vem recebendo na computação em nuvem, devido a: aumento dos custos das instalações, maior disponibilidade de recursos, aumento de demanda por aplicações de aprendizado de máquina e demanda por recursos especiais.

A próxima palestrante foi  a Professora Doutora Alba C. M. A. de Melo da Universidade de Brasília (UnB), a qual começou abordando o histórico de evolução dos supercomputadores a partir de 1968. A partir disso, a oradora citou aplicações que fazem uso de HPC, como aplicações biológicas: alinhamento pairwise (DNA, RNA e proteína) e alinhamento e dobramento pairwise (RNA).

Na sequência, a oradora elencou os principais serviços de HPC em nuvem: Amazon EC2 (AWS), Google GCP e Microsoft Azure. Neste contexto, foram listados os principais desafios dos provedores: consumo de energia (necessidade de investimentos em energia limpa e algoritmos para reduzir o consumo) e recursos ociosos (diferenças entre modelos spot e preemptive e escalonamento). Assim, também foram elencados desafios para o cliente: federação (como escolher o provedor), seleção (qual modelo de precificação e instância) e escalonamento eficiente de recursos.
O último palestrante a falar foi o Professor Doutor Philippe Navaux da Universidade Federal do Rio Grande do Sul (UFRGS), o qual iniciou a apresentação abordando os desafios de migração de clusters para cloud, como: portabilidade (linguagem de programação e plataforma, domínio específico e requer alto conhecimento da aplicação), custos (número de instâncias de processamento e tempo aceitável para a solução) e recursos (rede, armazenamento e processamento).

Na sequência o orador apresentou casos reais de migração, como o modelo climático BRAMS (Brazilian Regional Atmospheric Modeling System Mesoscale), o qual possuía mais de 30000 (trinta mil) linhas de código e exigiu meses de trabalho para a finalização. O sistema foi migrado para a Azure, sendo que o orador listou os seguintes itens positivos: alta escalabilidade, replicação e fácil de configurar. Já os pontos negativos citados foram: sem permissões de arquivo, limitações e perda de metadados.

Por fim, os oradores responderam questionamentos acerca do conteúdo apresentado, como: comparação da HPC com computação quântica, mecanismos de tolerância a falhas, energias renováveis e segurança dos dados. Assim, a palestra foi encerrada com os oradores ressaltando a importância dessa área de pesquisa, o desenvolvimento exponencial das tecnologias e as direções futuras.
