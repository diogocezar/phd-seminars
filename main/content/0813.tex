\section{Aplicações e Desafios de Pesquisa na Computação em Nuvem e Alto Desempenho}

\begin{center}
  \vspace{1cm}
  13 de agosto às 18H00m, Aplicações e Desafios de Pesquisa na Computação em Nuvem e Alto Desempenho, Painel 4 UFF, http://ev-ppgc.ic.uff.br/2020-2/paineis.html
  \vspace{1cm}
\end{center}

A professora doutora Lúcia Drummond da Universidade Federal Fluminense (UFF), iniciou definindo computação em nuvem como: paradigma de computação distribuída de larga escala, onde recursos computacionais estão disponíveis aos usuários através da Internet. Além disso, a palestrante também elencou os tipos de serviços em nuvem: Infrastructure as a Service (IaaS), Software as a Service (SaaS), Platform as a Service (PaaS) e Function as a Service (FaaS).

A professora Lúcia explicou sobre aplicações que demandam uso de supercomputadores, como: mecânica quântica, previsão de tempo, exploração de óleo e gás, modelagem molecular e simulações físicas. Destacou a a importância que HPC (High Performance Computing) vem recebendo na computação em nuvem. 

Em seguida, a fala foi da professora doutora Alba C. M. A. de Melo da Universidade de Brasília (UnB), ela falou sobre o histórico de evolução dos supercomputadores a partir de 1968 e citou aplicações que fazem uso de HPC, como aplicações biológicas. 

A palestrante mencionou os principais serviços de HPC em nuvem: Amazon EC2 (AWS), Google GCP e Microsoft Azure. Também foram listados os principais desafios dos provedores (consumo de energia  e recursos ociosos) e elencados desafios para o cliente (federação, seleção e escalonamento eficiente de recursos). 

O professor doutor Philippe Navaux da Universidade Federal do Rio Grande do Sul (UFRGS), palestrou em seguida, no qual falou sobre os desafios de migração de clusters para cloud, como: portabilidade, custos e recursos. Além disso, apresentou casos reais de migração, como o modelo climático BRAMS (Brazilian Regional Atmospheric Modeling System Mesoscale). 

Por último, os palestrantes responderam questionamentos acerca do conteúdo apresentado e em seguida salientaram a importância dessa área de pesquisa. 