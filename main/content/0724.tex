\section{Princípios da Educação Online}

\begin{center}
  \vspace{1cm}
  24 de julho às 13h30m, Princípios da Educação Online, Mariano Pimentel, Professor na UNIRIO e editor da coluna Educação da SBC Horizontes. Transmissão via Canal Youtube da Sociedade Brasileira de Computação: https://www.youtube.com/watch?v=BDUJvwlrUJY
  \vspace{1cm}
\end{center}

Inicia-se a apresentação destacando o grande aumento no acesso aos sistemas de edução a distância e online. Apresenta as direfenças entre as abordadens que podem ser praticadas no ensino a distância. Apresenta-se os conceitos da educação a distância, é uma modalidade educacional alternativa à educação presencial, e da educação online, que se enquadra em uma abordagem didático-pedagógica.

Na educação a distância os computadores conectados a internet são usados para compartilhamento de conteúdo, além de poder ajudar na correção automaticamente as respostas dos alunos, ou até mesmo recomendar o estudo de novos conteúdos em função do desempenho de cada aluno. Ou seja, propõe-se que o aluno estude sozinho e o computador seja utilizado como uma ``máquina de ensinar''. Essa abordagem é caracterizada como instrucionista-massiva, na qual os computadores representam uma evolução das mídias, mas não modificam o modelo de comunicação de massa, sendo este predominantemente unidirecional.
  
Propõe-se como alternativa à educação a distância instrucionista-massiva, a educação online, é formada por um conjunto de ações de ensino-aprendizagem ou atos de currículo, mediadas por interfaces digitais que potencializam práticas comunicacionais interativas e hipertextuais. 

São propostos oito princípios para guiar a adoção desta prática de ensino, sendo:

\begin{enumerate}
  \item Conhecimento como ``obra aberta'': busca a interferência, ressignificação, cocriação e completação. Tenta-se aplica evitar a ideia de conhecimento ``pronto'', no qual os alunos não possam questionar os conteúdos;
  \item Curadoria de conteúdos e síntese e roteiros de estudo: Busca-se colocar o foco em conteúdo em forma hipermídia, utilizando por exemplo: páginas da Wikipédia, página do Facebook, vídeos, apresentações, matéria de um blog, artigo científico, grupo no WhatsApp, entre outros. Busca-se que o docente ressalte os conteúdos de maior importância e não apresente texto especificamente criados para EAD;
  \item Ambientes computacionais diversas: não se restringindo aos ambientes EAD;
  \item Aprendizagem colaborativa: o professor participa das interações entre os próprios alunos (como um mediador) valorizando os múltiplos saberes e a colaboração entre os próprioa alunos;
  \item Conversação entre todos, em interatividade: disrupção do modelo onde apeans o professor fala e os alunos ouvem, para técnicas que envolvam por exemplo: bate-papo online (chat), reuniões por videoconferência, atendimento individualizado por redes sociais ou e-mail;
  \item Atividades autorais inspiradas nas práticas da cibercultura: propõe o método de aprendizado através da prática, levando o aluno a aplicar e transformar os conhecimentos da disciplina, ressignificando-os;
  \item Mediação docente online para colaboração: o professor desempenha um papel dinamizador no grupo;
  \item Avaliação formativa e colaborativa, baseada em competências: em vez de apenas exames presenciais;
\end{enumerate}

Para concluir explora-se o importante alcance do artigo publicado na revista SBC Horizontes, que teve mais de 40 mil visualizações, demonstrando a importância e a busca pelo tema abordado.
