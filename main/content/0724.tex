\section{Princípios da Educação Online}

\begin{center}
  \vspace{1cm}
  24 de julho às 13h30m, Princípios da Educação Online, Mariano Pimentel, Professor na UNIRIO e editor da coluna Educação da SBC Horizontes. Transmissão via Canal Youtube da Sociedade Brasileira de Computação: https://www.youtube.com/watch?v=BDUJvwlrUJY
  \vspace{1cm}
\end{center}

Inicia-se a apresentação destacando o grande aumento no acesso aos sistemas de edução a distância e online. Apresenta as direfenças entre as abordadens que podem ser praticadas no ensino a distância. Apresenta-se os conceitos da educação a distância, é uma modalidade educacional alternativa à educação presencial, e da educação online, que se enquadra em uma abordagem didático-pedagógica.

Na educação a distância os computadores conectados a internet são usados para compartilhamento de conteúdo, além de poder ajudar na correção automaticamente as respostas dos alunos, ou até mesmo recomendar o estudo de novos conteúdos em função do desempenho de cada aluno. Ou seja, propõe-se que o aluno estude sozinho e o computador seja utilizado como uma ``máquina de ensinar''. Essa abordagem é caracterizada como instrucionista-massiva, na qual os computadores representam uma evolução das mídias, mas não modificam o modelo de comunicação de massa, sendo este predominantemente unidirecional.
  
Propõe-se como alternativa à educação a distância instrucionista-massiva, a educação online, é formada por um conjunto de ações de ensino-aprendizagem ou atos de currículo, mediadas por interfaces digitais que potencializam práticas comunicacionais interativas e hipertextuais. 

São propostos oito princípios para guiar a adoção desta prática de ensino, sendo:

\begin{enumerate}
  \item Conhecimento como ``obra aberta'': em construção contínua, que busca a ressignificação, interferência, completação, cocriação. A ideia é evitar a percepção do conhecimento como um produto acabado, na qual os alunos não possam questionar os conteúdos;
  \item Curadoria de conteúdos + síntese e roteiros de estudo: em vez da produção de conteúdos próprios para EAD. Busca-se uma evolução na apresentação dos conteúdos em forma de hipermídia: vídeos, páginas da Wikipédia, apresentação no SlideShare, matéria de um blog, artigo científico, página do Facebook, grupo no WhatsApp, etc. Ou seja, a estratégia é que o docente ressalte os conteúdos de maior importância e não apresente texto específicos para EAD;
  \item Ambiências computacionais diversas: em vez de se restringir aos serviços do Ambiente de Aprendizagem;
  \item Aprendizagem em rede, colaborativa: em vez de aprendizagem solo. Neste conceito, o professor participa das interações entre os alunos, em forma de mediador, valorizando os múltiplos saberes;
  \item Conversação entre todos, em interatividade: em vez de apresentação de conteúdos. Transpor a abordagem padrão em que o professor fala e os alunos escutam, possibilitando o uso de diversas ferramentas em conjunto, como: reuniões por videoconferência de forma síncrona, bate-papo online (chat), atendimento individualizado por mensagem instantânea e e-mail.
  \item Atividades autorais inspiradas nas práticas da cibercultura: em vez de “estudo dirigido”. Baseia-se no método de aprendizado através da prática, levando o aluno a aplicar e transformar os conhecimentos da disciplina, ressignificando-os.
  \item Mediação docente online para colaboração: em vez de “tutoria reativa”. Busca-se uma mediação ativa, em que o professor desempenha um papel dinamizador no grupo.
  \item Avaliação formativa e colaborativa, baseada em competências: em vez de apenas exames presenciais.
	Por fim, o orador encerrou a apresentação ressaltando o alcance do artigo com este conteúdo publicado na revista SBC Horizontes, o qual teve mais de 40 mil visualizações, demonstrando a importância do tema e o interesse da comunidade.
\end{enumerate}