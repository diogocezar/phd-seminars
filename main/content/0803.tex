\section{Design Socialmente Consciente de Sistemas Computacionais}

\begin{center}
  \vspace{1cm}
  03 de agosto às 10h30m, Design Socialmente Consciente de Sistemas Computacionais, Cecilia Baranauskas, IC/UNICAMP. https://bbb.c3sl.ufpr.br/b/lui-gv2-9xx
  \vspace{1cm}
\end{center}

A oradora começou abordando as implicações sociais da tecnologia computacional, chamando a atenção para o lado humano e os dilemas éticos e sociais que envolvem a construção de um sistema. E, na sequência, trouxe informações a respeito dos principais cientistas da computação e as áreas de trabalho e pesquisa destes:

\begin{itemize}
  \item Kristen Nigaard: criador da programação Orientada a Objetos e pioneiro do design participativo. Ele incluiu na computação os aspectos sociais e políticos, além dos técnicos.
  \item Peter Naur: criou o ALGOL 60 e pesquisava o lado humano na programação e desenvolvimento de sistemas.
  \item Christiane Floyd: pioneira do design de software participativo evolucionário (precursora do desenvolvimento de software de código aberto). O foco de pesquisa dela é voltado para a área de paradigmas em Engenharia de Software e necessidade de múltiplas perspectivas na área.
  \item Ronald Stamper: pesquisa semiótica organizacional e fundamentos de sistemas de informação.
  \item Terry Winograd: trouxe novo entendimento para o design de ferramentas computacionais adequadas a propósitos humanos.
\end{itemize}

Na sequência a oradora chamou a atenção para a evolução da computação ao longo dos anos, saindo de tarefas humanas automatizadas e dos mainframes para a computação ubíqua (computação onipresente na vida humana). Assim, listou os principais desafios do presente e para o futuro, como: interdependência entre humanos e máquinas, coleta de dados (individual, casas inteligentes, cidades inteligentes, quantificação de si, etc.), armazenamento de dados combinados com algoritmos sofisticados de aprendizado de máquina e nível de “inteligência” no artificial e quem deveria ser responsabilizado.

Acerca da questão da responsabilidade, a oradora discorreu sobre dilemas e responsabilidades éticas que envolvem a construção e manutenção de sistemas computacionais, apresentando um exemplo de experimento: o “problema do bonde”. Em torno disso, a oradora ressaltou que não é possível que um sistema resolve dilemas éticos em todos os casos e em todos os lugares/países, devido principalmente a diferenças culturais e de legislação.
Na sequência foi abordada a projeção e criação de sistemas, na qual foi ressaltada a importância do envolvimento de todos (usuários e engenheiros de software) no processo, coexistindo assim, na interação do processo elementos informais, formais e técnicos. Há assim, a possibilidade de criar sistemas criativos e com um envolvimento responsável no processo.

Neste contexto, o designer deve apenas ser um facilitador no processo compartilhado de criação e as pessoas devem estar ativamente envolvidas tendo influência mútua. Desta forma, o processo permite o senso de criação de artefato, possibilitando a co-autoria do produto em criação.

A oradora também apresentou a plataforma “OpenDesign”, na qual técnicas e artefatos socialmente conscientes de sistemas computacionais estão disponíveis, sendo inspirada no fenômeno do código aberto.
Também foi discorrido sobre os sistemas socio(enativos), nos quais a ação é guiada pela percepção e estruturas cognitivas emergem a partir dos padrões sensores e motores para permitir a ação guiada pela percepção. A partir da apresentação do conceito, a oradora apresentou 3 experimentos realizados: Museu Exploratório de Ciências, Hospital Sobrapar e o Portal DEdIC (Unicamp).

Por fim, ressaltou a presença de novas tecnologias (interfaces tangíveis, vestíveis e naturais) e novas formas de interação no contexto da computação ubíqua e pervasiva e do futuro apresentam desafios que exigem uma nova mentalidade.
