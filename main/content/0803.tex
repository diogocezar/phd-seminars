\section{Design Socialmente Consciente de Sistemas Computacionais}

\begin{center}
  \vspace{1cm}
  03 de agosto às 10h30m, Design Socialmente Consciente de Sistemas Computacionais, Cecilia Baranauskas, IC/UNICAMP. https://bbb.c3sl.ufpr.br/b/lui-gv2-9xx
  \vspace{1cm}
\end{center}

Inicia-se a apresentação explorando as implicações sociais da tecnologia computacional, com o foco no lado humano e nos dilemas éticos e sociais que envolvem o desenvolvimento de um sistema computacional. Explora-se na sequência informações relacionadas aos principais cientistas da computação e de suas áreas de pesquisa:

\begin{itemize}
  \item Peter Naur: criou o ALGOL 60 e trabalhada com o lado humano na programação e desenvolvimento de sistemas;
  \item Ronald Stamper: pesquisa semiótica organizacional e fundamentos de sistemas de informação;
  \item Kristen Nigaard: criador da programação Orientada a Objetos e pioneiro do design participativo;
  \item Terry Winograd: trouxe novo entendimento para o design de ferramentas computacionais adequadas a propósitos humanos;
  \item Christiane Floyd: pioneira do design de software participativo evolucionário (precursora do desenvolvimento de software de código aberto).
\end{itemize}

Destaca-se na sequência a evolução do âmbito computacional ao longo dos anos, tornando tarefas humanas em tarefas automatizadas, além dos mainframes que possibilitam a tecnologia onipresente na vida humana. Explora-se então os principais desafios para o futuro, como por exemplo: interdependência entre humanos e computadores, coleta de dados (aparelhos eletrônicos, casas inteligentes, cidades inteligentes, etc.), armazenamento de dados combinados com algorítmos de aprendizado de máquina.

A responsabilidade ética também foi um tempo abordado, no que diz respeito a desenvolviemtno e manutenção de sistemas computacionais. Explora-se também que não é possível que um sistema resolva dilemas éticos com corretudo em todos os casos e em todos os lugares/países, devido principalmente a diferenças culturais e de legislação.

Foi ainda explorada a a projeção e criação de sistemas, ressantando a importância do envolvimento de todos (usuários e engenheiros de software) no processo. Permitindo o desenvolvimento de sistemas criativos e com um envolvimento responsável, colaborativo e diverso no processo de criação. Neste contexto, o designer deveser um facilitador no processo compartilhado de criação. Desta forma, o processo permite o senso de criação de artefato, possibilitando a co-autoria do produto em criação.

Explora-se ainda uma plataforma conhecida como ``OpenDesign'', na qual disponibiliza-se técnicas e artefatos de design socialmente conscientes. Essa iniciativa foi inspirada nos conceitos de Open Source.

Além disso, discorre-se sobre sistemas socio(enativmos) nos quais a ação é guiada pela percepção e estruturas cognitivas emergem a partir dos padrões sensores e motores para permitir a ação guiada pela percepção. A partir da apresentação do conceito, a oradora apresentou 3 experimentos realizados: Museu Exploratório de Ciências, Hospital Sobrapar e o Portal DEdIC (Unicamp).

Finalizando a apresentação, ressalta-se a que a preença de novas tecnologias (interfaces tangíveis, vestíveis e naturais) e novas formas de interação no contexto da computação ubíqua e pervasiva e do futuro são desafios que exigem uma nova mentalidade por parte de todos os colaboradores que criam novas tenologias.
