\section{A Inovação em Computação Aplicada à Indústria}

\begin{center}
  \vspace{1cm}
  11 de agosto às 18H00m, A Inovação em Computação Aplicada à Indústria, Painel 2 UFF, http://ev-ppgc.ic.uff.br/2020-2/paineis.html
  \vspace{1cm}
\end{center}

Nesta palestra, houve a participação do empresário Marcelo Sales da CyberLabs e do Doutor Pedro M. Cruz e Silva da NVIDIA.

A palestra contou com a apresentação da NVIDIA, e como seu processamento gráfico em jogos e em vídeos evoluíram. Houve também o destaque para os supercomputadores e a pesquisa em algoritmos de Deep Learning e inteligência artificial incluindo a robótica.

Houve também a abordagem da atuação da empresa em combate à Covid 19, através da análise de dados em tempo real, classificação de imagens e sequenciamento do genoma viral, entre outros. 
O empresário Marcelo Sales, apresentou a empresa CyberLabs, e seus produtos que se utilizam de algoritmos de inteligência artificial, como o InSigntNow, no qual analisa imagens de vídeo em tempo real, proporcionando a contagem de pessoas, e determinando o que está acontecendo e o que ainda vai acontecer, e vem sendo utilizados em várias cidades do pais. 
O outro produto apresentado é o Keyapp, que faz reconhecimento facial e é utilizado para sistemas de segurança aplicados. 

Na palestra foi comentada a relação ética da aplicação de algoritmos, que foi apontado um assunto de ampla discussão, para que haja a responsabilidade para cada método computacionais e sua forma de ser feito.

Os oradores responderam algumas dúvidas, e no final apontaram a importância da área da inteligência artificial, e um desenvolvimento promissor nos próximos anos, comtemplando a experiência vivida com o Covid, que mostra como a computação pode servir para ajudar em várias áreas de problemáticas sociais.