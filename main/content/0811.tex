\section{A Inovação em Computação Aplicada à Indústria}

\begin{center}
  \vspace{1cm}
  11 de agosto às 18H00m, A Inovação em Computação Aplicada à Indústria, Painel 2 UFF, http://ev-ppgc.ic.uff.br/2020-2/paineis.html
  \vspace{1cm}
\end{center}

A palestra contou com a participação do empresário Marcelo Sales da CyberLabs e do Doutor Pedro M. Cruz e Silva da NVIDIA. O gerente de arquitetura de soluções da NVIDIA iniciou a apresentação falando sobre as áreas de atuação e a evolução da NVIDIA ao longo dos anos. Destacou principalmente o processamento gráfico (em jogos e filmes), os supercomputadores que a NVIDIA possui para pesquisa em algoritmos de deep learning, e Inteligência Artificial (treinamento, inferência e robótica). Assim, o orador demonstrou a ampla atuação da NVIDIA, saindo do nicho de mercado voltado para jogos e atualmente atuando em soluções para diversos mercados.

Na sequência foi abordada a atuação da NVIDIA no combate ao COVID-19, através de: análise de dados em tempo real, classificação de imagens, sequenciamento do genoma viral, etc. Assim, o orador continuou apresentando os resultados recentes obtidos em deep learning utilizando GPUs, inclusive, ressaltando a superação de algoritmos tradicionais de visão computacional. Adicionalmente, o orador trouxe o exemplo do método BERT que foi treinado com a base de dados da Wikipedia e reportou os resultados do algoritmo que superou um humano para responder questões. Na sequência o orador ressaltou o aumento da complexidade de modelos de redes neurais, saindo dos 340 milhões parâmetros do método BERT para mais de 2 bilhões para o projeto Megatron.

Em uma segunda parte, o empresário Marcelo Sales começou apresentando a empresa CyberLabs e os produtos em desenvolvimento que utilizam algoritmos de Inteligência Artificial. O primeiro produto comentado foi o InSightNow, o qual faz análise de imagens de vídeo em tempo real, permitindo contagem de pessoas, determinar o que está acontecendo ou vai acontecer. Ainda ressaltou que o produto já é utilizado em várias cidades do Brasil.
Outro produto apresentado foi o KeyApp, o qual faz reconhecimento facial e possibilita controle de acesso a sistemas de segurança integrados. Acerca disso, na sequência surgiram dúvidas em relação a questão ética da aplicação dos algoritmos, sendo que o orador indicou que há necessidade de uma ampla discussão na sociedade para determinar as responsabilidades de cada um na concepção dos métodos computacionais.

Por fim, os oradores ainda responderam questões acerca de tecnologias da NVIDIA, parcerias da CyberLabs com as Universidades do Brasil, desafios no desenvolvimento dos métodos em relação a fake news e a relação interdisciplinar da Inteligência Artificial nas áreas jurídicas, política e indústria. Assim, a palestra foi finalizada com os oradores ressaltando a importância da área de Inteligência Artificial e o futuro promissor para a área, na qual espera-se um desenvolvimento exponencial nos próximos anos, tomando como base os últimos tempos e principalmente nessa época de pandemia, que demonstrou a importância da computação para auxiliar na resolução de vários problemas da sociedade.
