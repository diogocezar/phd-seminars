\section{Excelência em Pesquisa em Computação}

\begin{center}
  \vspace{1cm}
  10 de agosto às 18H00m, Excelência em Pesquisa em Computação, Painel 1 UFF, http://ev-ppgc.ic.uff.br/2020-2/paineis.html
  \vspace{1cm}
\end{center}

O Professor Doutor da Universidade Federal Fluminense (UFF), Célio Albuquerque, começou a apresentação explicando o PROEX (CAPES): Programa de Excelência Acadêmica. No PROEX, os conceitos 6 e 7 são considerados de excelência internacional, desempenho equivalente a dos centros internacionais de excelência na área, nível diferenciado em relação aos demais programas da área, solidariedade e nucleação e impacto na sociedade.

Na sequência apresentou características da UFF: 60 anos de história, 3.200 professores, 66.000 estudantes e 10 campus. E o Instituto de Computação da UFF possui: 3 prédios próprios, 65 professores, 500 alunos de ciência da computação e 500 de sistemas de informação e mais de 270 alunos no programa de pós-graduação, sendo que já formou mais de 700 mestres e doutores. A partir dos números de egressos apresentados, o orador detalhou a ocupação atual dos ex-alunos, sendo que a maioria dos mestres formados (55\%) trabalham em empresas e os doutores trabalham em sua maioria em universidades e institutos.

O orador continuou a apresentação ressaltando o impacto da UFF na sociedade: projetos P\&D (ANP, ANEEL), parceria NVIDIA (super computador DGX-1) e computação aplicada à indústria do petróleo. Além disso, também trouxe informações acerca da internacionalização da instituição: projetos com EUA, UK e França; 10\% do corpo docente realiza estágio pós-doutoral no exterior; atração de professores de renome internacional; organização de eventos internacionais (SIBGRAPI 2017, IPDPS 2019, LANOMS 2019, IWSSIP 2020); participação em comitê de programa de conferências e corpo editorial de periódicos internacionais; prêmios internacionais (VectorizeMove), Google Research Award e Microsoft Latin America PhD Awards.

A próxima Professora a falar foi a Doutora da Universidade Federal do Rio Grande do Sul (UFRGS), Luciana Buriol, a qual começou a apresentação elencando dados da UFRGS: programa de mestrado (criado em 1973) e doutorado (1989), nota 7 (desde 2013), 54 docentes no programa de pós-graduação em computação (mais 3 colaboradores), 96 doutorandos e 198 mestrandos (com 365 doutores formados e 1.659 mestres), 7 áreas de concentração (divididas em 18 linhas de pesquisa).

Na sequência, a oradora destacou os casos de sucesso da UFRGS: ranking internacional AI 2000, Aegro, COVID-19 Analysis Tools, docente colaborador na UNESCO e na ONU. Além disso, ressaltou a produção científica de impacto da UFRGS (1ª na América do Sul) e participação em comitês editoriais (20 periódicos internacionais).

O terceiro orador, Artur Zivani, do Laboratório Nacional de Computação Científica (LNCC), começou apresentando a estrutura do LNCC: 50 pesquisadores, conceito 6 pela CAPES e faz parte do Ministério da Ciência, Tecnologia, Inovação (MCTI). Além disso, possuem o supercomputador Santos Dumont (top 500 mundial).

O orador prosseguiu detalhando números históricos do LNCC: 3 prêmios CAPES de melhor tese e 3 menções honrosas, 49 doutorandos e 29 mestrandos, e 140 doutores e 173 mestres formados. Destes egressos, 79 estão no setor público acadêmico, 5 não acadêmico, 4 setor privado acadêmico, 4 privado, 20 em pós-doutorado no país, 5 no exterior e 5 em atividades acadêmicas no exterior. O corpo docente do LNCC é formado por 41 pessoas, sendo 29 membros permanentes e 12 colaboradores.

Por fim, o orador finalizou a apresentação destacando a inserção nacional e internacional do LNCC: articulação com empresas de petróleo, publicações docentes em veículos importantes, 81 artigos em periódicos, 86 trabalhos completos em anais de congressos e 22 livros, coordenação e participação em 23 projetos internacionais em 2019, vários docentes participam do Editorial Boards de periódicos importantes e comitês de avaliação internacional, e mais recentemente, o LNCC fez um acordo com o INRIA (França) para cooperação nas áreas de Big Data e Inteligência Artificial.
