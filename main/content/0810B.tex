\section{Excelência em Pesquisa em Computação}

\begin{center}
  \vspace{1cm}
  10 de agosto às 18H00m, Excelência em Pesquisa em Computação, Painel 1 UFF, http://ev-ppgc.ic.uff.br/2020-2/paineis.html
  \vspace{1cm}
\end{center}


Nesta palestra, o Professor Doutor pela UFF (Universidade Federal Fluminense), Célio Albuquerque, se colocou a explicar o PROEX (programa de excelência acadêmica). O mesmo apresentou a UFF, dizendo dos seus 60 anos de história, e contabilizando 3200 professores, 66.000 estudantes, 500 alunos de ciência da computação, 500 alunos de sistemas de informação e mais 270 alunos no programa de pós graduação que já formou 700 alunos mestres e doutores. O Professor Doutor também informou que dentre os egressos, aqueles que fizeram mestrado, 55\% trabalham em empresas e os doutores formados, trabalham em universidades. 

A UFF, tem grande importância na sociedade por conta de alguns projetos apresentados, como: parceria NVIDIA (super computador DGX-1) e projeto P\&D (ANP, ANEEL). Há também a apresentação de informações de projetos internacionais da instituição de ensino, nos países EUA, França e UK, onde 10\% do corpo docente da Universidade faz estágio pós-doutoral no exterior. Há ainda a participação em comitês de programas de conferencias, entre outros.

A professora Doutora da UFRGS, Lucia Buriol, apresentou dados do programa de pos graduação da universidade, mostrando que o mestrado foi criado em 1973, o doutorado em 1989, e mantem a nota 7 desde do anos de 2013, contando com 54 docentes no programa de pós graduação em computação, 96 doutorandos, e 198 mestrando, entre eles com 365 doutores formados e 1659 mestres, tendo 7 áreas diferentes de concentração, e dividida em 18 linhas de pesquisa. Contudo, a doutora, continua destacando os casos de sucesso de sua Universidade.

E por fim o orador Artur Zivani, do laboratório nacional de computação cientifica (LNCC), apresentou sua instituição, contabilizando 50 pesquisadores, conceito 6 pela CAPES que fazem parte do Ministerio da Ciencia, Tecnologia, Inovação  (MCTI). Ainda possuem o supercomputador Santos Dumont (top 500 mundial). O orador continuou mostrando os aspectos importantes da instituição como 3 premios CAPES de melhor tese e 3 mencoes honrosas, 29 mestrandos e 49 doutorandos. E a equipe do corpo docente do LNCC  e formado por 41 pessoas, sendo 29 membros permanentes e 12 colaboradores. Contudo o orador destacou a inserção nacional e internacionais do LCNN, que está articulado com empresas de petróleo, e publicações docentes em veículos importantes para a área, e também destacando o acordo feito com o INRIA na França para a cooperação nas Áreas de inteligência artificial.