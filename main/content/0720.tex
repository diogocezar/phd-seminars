\section{Information Ethics as an Engineering Discipline}

\begin{center}
  \vspace{1cm}
  20 de julho às 10h30m, Ethics and STICs,  Thomas Baudel, IBM Research – Franca, https://bbb.c3sl.ufpr.br/b/lui-gv2-9xx
  \vspace{1cm}
\end{center}

A apresentação se inicia com o levantamento da questão ``O que é uma decisão de Engenharia?''. A discussão segue com foco nos possíveis processos aplicados ao desenvolvimento de sistemas computacionais. Argumenta-se que na prática os processos são por essência bastante simples e não existem um grande poder de processamento. O problema apresentado se apresenta nas questões de manutenabilidade, causado na maioria das vezes por razões humanas, argumentando-se que o conhecimento sobre como as funções lógicas de processo/decisão não são universais como os processos são. Isso decorre da falta de contexto, dificuldade na manipulação de casos especiais e fragmentação de tarefas.

Na sequência, apresentou-se novas explorações para decisões de engenharia. Duas opções mais atuais foram detalhadas. A \textit{Pulsar} coloca o usuário de volta no controle, permitindo customizações locais dos processos e propõe técnias de compartilhamento e reutilização. Esta opção utiliza um catálogo de ``habilidades'' que permite o compartilhamento de conhecimento do negócio, com isso é possível introduzir flexibilidade no processo de projeto, compartilhar os melhores exemplos de boas práticas, compartilhar conhecimento e promover um sistema de melhoria contínua. Os usuários são vistos como \textit{designers}. Já a \textit{AIDA} (Artificial Intelligence for Decision Automation), aumenta o uso de memória artificial no tomador de decisão e inferência, desta forma, garante-se que as mesmas causas produzem os mesmos efeitos, aproveitando o histórico de decisões passadas, apoiando os usuários nas tomadas de decisões. Isso é feito com  o treinamento de classificadores baseados em instâncias passadas de aprovações, e propõe a recomendação do classificador como uma sugestão para o usuário. As recomendações podem ser por: vizinhos mais próximos, árvore de decisão, incluir nível de confiança ou incluir ``explanações''. Um exemplo em que a AIDA poderia ser aplicada: tarefa de aprovar empréstimo (``Baseado em decisões passadas nós recomendamos aprovar o empréstimo com um grau de confiança de 95\%'').

Na sequência, argumenta-se ainda sobre as questões reais de engenharia: responsabilidade, transparência, viés (cultural, cognitivo, estatístico, entre outros...), apropriação de tecnologia, escolhas de sociedade. Essas questões levam a dilemas éticos, que recaem sobre os profissionais do âmbito computacional, quando estes quase sempre estão focados apenas entregar uma funcionalidade.

Na exploração destes dilemas éticos, são consideradas situações que se contradizem. Por exemplo: a) aos autores são garantidos os direitos sobre a criação original deles; b) o compartilhamento gratuito de coisas imateriais é um benefício da sociedade da internet. Isso leva a seguinte pergunta: fazer o que é bom, ou fazer o que é certo? Pois a questão ética envolve diversos temas, como: Lei, moralidade, integridade científica, opinião pessoal contra construção de um consenso. Assim, um conjunto de análises éticas possibilita decisões concretas serem tomadas na presença de dilemas éticos.

Por fim, são apresentadas as categorias em que a área de pesquisa em Ética da Informação trabalha: impacto dos sistemas de computação, impacto social, econômico e humano, efeitos na natureza e impactos de transformação profunda (desmaterialização de si mesmo).
