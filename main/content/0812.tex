\section{Desafios da Computação no Cenário de Cidades Inteligentes}

\begin{center}
  \vspace{1cm}
  12 de agosto às 18H00m, Desafios da Computação no Cenário de Cidades Inteligentes, Painel 3 UFF, http://ev-ppgc.ic.uff.br/2020-2/paineis.html
  \vspace{1cm}
\end{center}

A professora doutora Flavia Bernardini e Flávia C. Delicato, que pertencem Universidade Federal Fluminense (UFF), iniciaram apresentando sobre a experiência com projetos de P\&D em Aprendizado de Máquina, Big Data e Cidades Inteligentes. Evidenciaram o fortalecimento do relacionamento com a prefeitura, a transparência passiva e muitos problemas que ainda não foram resolvidos.

Em seguida, alguns dados foram apresentados: nas próximas décadas mais da metade da população mundial viverá em cidades, a América Latina sendo capaz de atingir 80\%. Com este cenário é inevitário que alguma questões sejam discutidas: energia, transporte, prédios, infraestrutura, dentro outros.

Em 1997, surge o tema “Cidades Inteligentes”, porém este tema só começa a ter mais foco a partir de 2010. Em 2007, Giffinger e Gudrun, explicam cidade inteligente, no qual quando seus investimentos em capital humano e social e também em transporte urbano e infraestrutura de TIC, promovem o desenvolvimento econômico sustentável e a qualidade de vida, sabendo gerenciar os recursos naturais, por meio do governo participativo.

As plataformas de cidades inteligentes precisam atender a requisitos funcionais e não funcionais. Requisitos funcionais: gestão dos dados, gestão dos sensores, processamento dos dados, acesso aos dados da plataforma, gerenciamento de recursos, ambientes para o desenvolvimento de aplicações para a cidade, ambientes para o desenvolvimento de aplicações para a cidade e ambientes para execução de aplicações. Requisitos não funcionais: interoperabilidade, escalabilidade, elasticidade, adaptabilidade e ciência de contexto, segurança e privacidade. 
O Middleware é uma camada de software inserida nas aplicações de cidades inteligentes. Ela oferece soluções para enfrentar a heterogeneidade em sistemas distribuídos, sendo necessário promover interoperabilidade.

A palestra continuou com a professora doutora Thais Batista da Universidade Federal do Rio Grande do Norte (UFRN), destacando a sinergia necessária entre tecnologias, conhecimento e habilidades para possibilitar cidades inteligentes. Comentou também sobre a interoperabilidade entre sensores de múltiplos fornecedores, sistemas em diferentes linguagens de programação e dados em diferentes formatos.

Além disso, abordou questões que envolvem cidades inteligentes, tais como: integração, big data, adaptação dinâmica, escalabilidade e segurança. Apresentou o SGeoL, um sistema de geoprocessamento que a UFRN desenvolveu em parceria com o governo estadual.

O professor doutor Luiz Satoru Ochi da UFF,  foi o últmo a palestrar, abordando o contexto de logística e transportes em cidades inteligentes. O professor discursou sobre exemplos reais de uso de drones para resolver problemas cotidianos. Por fim, os trabalhos resultaram, para o grupo de pesquisa da UFF, em várias participações em congressos internacionais e publicações. 
