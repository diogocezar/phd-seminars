\section{Desafios da Computação no Cenário de Cidades Inteligentes}

\begin{center}
  \vspace{1cm}
  12 de agosto às 18H00m, Desafios da Computação no Cenário de Cidades Inteligentes, Painel 3 UFF, http://ev-ppgc.ic.uff.br/2020-2/paineis.html
  \vspace{1cm}
\end{center}

As primeiras palestrantes a falarem sobre o tópico foram a Professora Doutora Flavia Bernardini e Flávia C. Delicato, ambas da Universidade Federal Fluminense (UFF), as quais discorreram sobre a experiência com projetos de P\&D em Aprendizado de Máquina, Big Data e Cidades Inteligentes (projeto piloto em Rio das Ostras). Ressaltaram o fortalecimento do relacionamento com a prefeitura, a transparência passiva por parte desta e muitos problemas que ainda não foram resolvidos.

Na sequência, foram apresentados alguns dados: nas próximas décadas mais da metade da população mundial viverá em cidades, sendo que a América Latina pode atingir 80\%. A partir dessa alta densidade demográfica surgem várias necessidades: energia, transporte, prédios, infraestrutura, etc. 

Neste contexto, o tema “Cidades Inteligentes” surge em 1997 e passa a receber mais foco a partir de 2010. Sendo que, em 2007, Giffinger e Gudrun, definiram cidade inteligente, quando seus investimentos em capital humano e social, em transporte urbano e infraestrutura de TIC alimentam o desenvolvimento econômico sustentável e uma melhor qualidade de vida, com sábio gerenciamento de recursos naturais, através do governo participativo.

Os requisitos que plataformas de cidades inteligentes devem atender:

Requisitos funcionais:

\begin{itemize}
  \item Gestão dos dados;
  \item Gestão dos sensores;
  \item Processamento dos dados;
  \item Acesso aos dados da plataforma;
  \item Gerenciamento de recursos;
  \item Ambientes para o desenvolvimento de aplicações para a cidade;
  \item Ambientes para execução de aplicações;
\end{itemize}

Requisitos não funcionais:

\begin{itemize}
  \item Interoperabilidade;
  \item Escalabilidade;
  \item Elasticidade, adaptabilidade e ciência de contexto;
  \item Segurança e privacidade;
\end{itemize}

Uma camada de software inserida nas aplicações de cidades inteligentes é o Middleware. Esta camada fornece soluções para lidar com heterogeneidade em sistemas distribuídos e é necessário para promover interoperabilidade.
Na sequência, a Professora Doutora Thais Batista da Universidade Federal do Rio Grande do Norte (UFRN), continuou a palestra ressaltando a sinergia necessária entre tecnologias, conhecimento e habilidades para possibilitar cidades inteligentes. Desta forma, abordou também a questão da interoperabilidade entre sensores de múltiplos fornecedores, sistemas em diferentes linguagens de programação e dados em diferentes formatos.

A oradora continuou abordando questões que envolvem cidades inteligentes, como: integração (necessidade de padrões mundiais), big data, adaptação dinâmica, escalabilidade e segurança. Ainda foi apresentado o SGeoL, um sistema de geoprocessamento que a UFRN desenvolveu em parceria com o governo estadual, o qual proporciona diversos serviços à população, como: camada de aglomeração de pessoas, educação, serviços diversos, etc.

Por fim, o último palestrante a falar foi o Professor Doutor Luiz Satoru Ochi da UFF, o qual abordou o contexto de logística e transportes em cidades inteligentes. Assim, o orador comentou sobre exemplos reais de uso de drones para resolver problemas cotidianos, como: inspeção e avaliação na cidade, entrega de produtos e vigilância em áreas específicas. Desta forma, os trabalhos resultaram em diversas participações em congressos internacionais e publicação de livros acerca do assunto por parte do grupo de pesquisa da UFF.